\section{Strategie-Einheit}

\begin{frame}
	\frametitle{Übersicht}
	
	\begin{itemize}
		\item Verantwortlich für alle Aktionen während dem Spiel
		\item Dynamische Jobauswahl
		\item Pfadsuche
		\item Ausweichmanöver bei Kollisionswarnung
		\item Kommunikation mit Partner
	\end{itemize}
	
\end{frame}

\begin{frame}
	\frametitle{Aufgaben (\textit{Jobs})}
	
	\begin{figure}
		\begin{tikzpicture} [scale=0.6, transform shape]
			\coordinate (strategySize) at (18.5,10);
			\def\top{9.4};
			\def\sep{0.3};
			\def\varHeight{0.8};
			\def\varWidth{4};
			\def\jobHeight{8.5};
			\def\jobWidth{13.3};
			\def\jobTop{8.3};
			\def\fsmHeight{\jobTop+3*\sep};
			\def\fsmWidth{\jobWidth-4*\sep-\varWidth};
			\def\fsmText{4.05};
			
			\coordinate (jobStart) at (\sep+\varWidth+\sep, \top-\sep);
			\draw (0,0) rectangle (strategySize) node[pos=0.5, shift={(0, 4.5)}]{\Large \textbf{strategy}}; %strategy box
			
			%strategy components
			\draw (\sep, \top-\sep-0*\sep-0*\varHeight) rectangle ++(\varWidth, -\varHeight) node[pos=0.5]{Status variables};
			\draw (\sep, \top-\sep-1*\sep-1*\varHeight) rectangle ++(\varWidth, -\varHeight) node[pos=0.5]{current Job};
			\draw (\sep, \top-\sep-1*\sep-2*\varHeight) [draw=none] rectangle ++(\varWidth, -\varHeight) node[pos=0.5]{$\vdots$};
			
			%jobs
			\draw (jobStart) [shift={(1*\sep, -1*\sep)}, fill=gray!10, visible on=<2->] rectangle ++(\jobWidth, -\jobHeight);
			\draw (jobStart) [shift={(0.5*\sep, -0.5*\sep)}, fill=gray!10, visible on=<2->] rectangle ++(\jobWidth, -\jobHeight);
			\draw (jobStart) [shift={(0*\sep, -0*\sep)}, fill=gray!10, visible on=<2->] rectangle ++(\jobWidth, -\jobHeight) node[pos=0.5, shift={(0, 0.45*\jobHeight)}] {\textbf{Job}};
			
			%job content
			\begin{scope} [shift={(\sep+\varWidth+\sep+\sep, \jobTop)}]
				\draw (0,-0*\varHeight-0*\sep) [fill=white, visible on=<3->] rectangle ++(\varWidth, -\varHeight) node[pos=0.5] {ID};
				
				\draw (0,-1*\varHeight-1*\sep) [fill=white, visible on=<3->] rectangle ++(\varWidth, -\varHeight) node[pos=0.5] {currentState};
				
				\draw (1*\sep,-2*\varHeight-3*\sep) [fill=white, visible on=<4->] rectangle ++(\varWidth, -\varHeight);
				\draw (0.5*\sep,-2*\varHeight-2.5*\sep) [fill=white, visible on=<4->] rectangle ++(\varWidth, -\varHeight);
				\draw (0,-2*\varHeight-2*\sep) [fill=white, visible on=<4->] rectangle ++(\varWidth, -\varHeight) node[pos=0.5] {requirementFunction()};
				
				\draw (1*\sep,-3*\varHeight-5*\sep) [fill=white, visible on=<4->] rectangle ++(\varWidth, -\varHeight);
				\draw (0.5*\sep,-3*\varHeight-4.5*\sep) [fill=white, visible on=<4->] rectangle ++(\varWidth, -\varHeight);
				\draw (0,-3*\varHeight-4*\sep) [fill=white, visible on=<4->] rectangle ++(\varWidth, -\varHeight) node[pos=0.5] {priorityFunction()};
				
				\draw (0,-4*\varHeight-6*\sep) [fill=white, visible on=<6->] rectangle ++(\varWidth, -\varHeight) node[pos=0.5] {Target};
				
				\draw (0,-5*\varHeight-6*\sep) [draw=none, visible on=<6->] rectangle ++(\varWidth, -\varHeight) node[pos=0.5]{$\vdots$};
				
				\begin{scope} [shift={(\varWidth+2*\sep, 0)}]
				
					\draw (0,0) [fill=white, visible on=<5->] rectangle ++(\fsmWidth, -\fsmHeight);
					\node [shift={(\fsmText, -0.4)}, visible on=<5->] {State Machine};
					\node [shift={(\fsmText, -0.8)}, anchor=north, visible on=<5->] {
						\footnotesize
						\begin{tabular}{|c|c|c|c|}
							\hline
							currentState & evaluation() & task() & nextState \\ \hline \hline
							& & & \\ \hline
							& & & \\ \hline
							& & & \\ \hline
							& & & \\ \hline
							& & & \\ \hline
							& & & \\ \hline
							& & & \\ \hline
							& & & \\ \hline
							& & & \\ \hline
							& & & \\ \hline
							& & & \\ \hline
							& & & \\ \hline
							& & & \\ \hline
							& & & \\ \hline
						\end{tabular}
					};
				
				\end{scope}
			
			\end{scope}
		
		\end{tikzpicture}
	\end{figure}
\end{frame}

\begin{frame}
	\frametitle{Pfadsuche und Spielfeld}
	\vspace{-1.5em}
	\only<1>{\begin{figure}\includegraphics[scale=0.035]{../images/spielfeld/playingAreaPlain.pdf}\end{figure}}	\only<2>{\begin{figure}\includegraphics[scale=0.035]{../images/spielfeld/playingAreaAreas.pdf}\end{figure}}	\only<3>{\begin{figure}\includegraphics[scale=0.035]{../images/spielfeld/playingAreaWaypoint.pdf}\end{figure}}	\only<4>{\begin{figure}\includegraphics[scale=0.035]{../images/spielfeld/playingAreaGraph.pdf}\end{figure}}
	\only<5>{\begin{figure}\includegraphics[scale=0.035]{../images/spielfeld/playingAreaPath1.pdf}\end{figure}}
	\only<6>{\begin{figure}\includegraphics[scale=0.035]{../images/spielfeld/playingAreaPath2.pdf}\end{figure}}
	\only<7>{\begin{figure}\includegraphics[scale=0.035]{../images/spielfeld/playingAreaPath3.pdf}\end{figure}}
	
\end{frame}

\begin{frame}
	\frametitle{\textit{collision handling}}
	\vspace{-1.5em}
	\only<1>{\begin{figure}\includegraphics[scale=0.035]{../images/spielfeld/playingAreaPath4.pdf}\end{figure}}
	\only<2>{\begin{figure}\includegraphics[scale=0.035]{../images/spielfeld/playingAreaPath5.pdf}\end{figure}}
	\only<3>{\begin{figure}\includegraphics[scale=0.035]{../images/spielfeld/playingAreaPath6.pdf}\end{figure}}
\end{frame}

\begin{frame}
	\frametitle{Erkentnisse und Ergebnis}
	
	\begin{itemize}
		\item Wiederverwendbarkeit der Jobs
		\item Komplexität
		\item Ausweichpfade nur sehr selten möglich
	\end{itemize}
	
\end{frame}