\section{Roboterfunktionen}
\subsection{Grosser Roboter}

\begin{frame}
	\frametitle{Supportrad}
	
\end{frame}

\begin{frame}
	\frametitle{Walze und Leitplanke}
	
\end{frame}

\begin{frame}
	\frametitle{Riemen und Abstreifvorrichtung}
	
\end{frame}

\begin{frame}
	\frametitle{Schussmechanismus}
	
\end{frame}

\subsection{Kleiner Roboter}

\begin{frame}
	\frametitle{Schere und Schranken}
	Mithilfe eines Scherenzugs sollten die \textit{Lunar Modules} aus den Raketen gezogen werden.
	Die Schranken verhindern ein umfallen der nachrutschenden \textit{Lunar Modules}.
	%Bilder
	Dieses Konzept funktionierte nicht wunschgemäss und wurde deshalb ersetzt.	
\end{frame}

\begin{frame}
	\frametitle{Greifer und Pressfinger}
	Greifer und Pressfinger wurden als Ersatz für Schere und Schranken entwickelt und konnten erfolgreich eingesetzt werden.
	%Bilder
\end{frame}

\begin{frame}
	\frametitle{Ring}
	Die \textit{Lunar Modules} haften während dem Transport an den Klebestellen des Rings.
	Durch drehen des Rings werden sie gekippt und abgestreift.
\end{frame}

\begin{frame}
	\frametitle{Drehmechanismus}
	Ein Arm mit einem Farbsensor und einem Rad kann ausgeklappt werden um \textit{Lunar Modules} zu drehen oder zu verschieben.
\end{frame}