\documentclass[aspectratio=169]{beamer}
\usepackage{german}
\usepackage[utf8]{inputenc} %for windows
\usepackage[T1]{fontenc}
\usepackage{textcomp}
\usepackage{gensymb}
\usepackage{graphicx}
\usepackage{tikz}
\usepackage{pgfplots}
\usepackage{xcolor}
\usepackage{siunitx}
\usepackage{listings}
\usepackage{caption, subcaption}
\usepackage{verbatim}
\usepackage{eso-pic} %to draw on top right corner
\usepackage{xpatch}

%define hsr colors
\definecolor{hsrBlue}{RGB}{000,101,163}
\definecolor{hsrHematite}{RGB}{110,028,080}
\definecolor{hsrLakeGreen}{RGB}{084,140,134}
\definecolor{hsrReed}{RGB}{123,105,081}
\definecolor{hsrPetrol}{RGB}{000,115,141}
\definecolor{hsrBasswood}{RGB}{186,189,093}
\definecolor{hsrGray}{RGB}{198,199,200}
\definecolor{hsrBlack}{RGB}{026,023,027}

%design & color
\usetheme[width=2.2cm]{PaloAlto}
\usecolortheme{beaver}
\useinnertheme{circles}
\usefonttheme[onlymath]{serif}
\setbeamercolor{section in sidebar}{fg=hsrBlack}
\setbeamercolor{itemize item}{fg=hsrBlue}
\setbeamercolor{itemize subitem}{fg=hsrLakeGreen}
\setbeamercolor{item projected}{fg=white,bg=hsrBlack}
\setbeamercolor{title in sidebar}{fg=hsrBlack}
\setbeamercolor{author in sidebar}{fg=hsrBlue} 
\setbeamercolor{caption name}{fg=hsrBlue}
\setbeamercolor*{title}{fg=hsrBlue}
\setbeamercolor{frametitle}{fg=hsrBlue, bg=hsrGray!15!white}
\setbeamercolor{sidebar}{bg=hsrGray!15!white}
\setbeamercolor{logo}{bg=hsrGray!0!white}

%disable navigation symbols
\beamertemplatenavigationsymbolsempty
%slide numbers
\setbeamertemplate{footline}[frame number]

%used for drawing n(r)-Area
\definecolor{lGray}{gray}{0.8}
\definecolor{llGray}{gray}{0.9}
\usepgfplotslibrary{fillbetween}
\usetikzlibrary{fadings}

\definecolor{listinggray}{gray}{0.9}
\definecolor{lbcolor}{rgb}{0.97,0.97,0.97}
\definecolor{lightGray}{gray}{0.1}

\definecolor{cOrange}{HTML}{996633}
\definecolor{clOrange}{HTML}{DBB48D}
\definecolor{cBlue}{HTML}{336699}
\definecolor{clBlue}{HTML}{A0BCD8}
\definecolor{cGreen}{HTML}{339966}
\definecolor{clGreen}{HTML}{94D4B4}
\definecolor{cRed}{HTML}{993333}
\definecolor{clRed}{HTML}{D0B0B0}
\definecolor{cGray}{gray}{0.4}
\definecolor{clGray}{gray}{0.96}

\tikzset{>=stealth}

\newcommand{\eurobot}{$\text{Eurobot}^\text{open}$\ }

% ----- lstListings (C++ code with syntax highlighting) -----
\lstset{
	backgroundcolor=\color{lbcolor},
	xleftmargin=0.5cm,
	xrightmargin=0.5cm,
	tabsize=2,
	language=C++,
	captionpos=b,
	frame=none,
	numbers=none,
	numberstyle=\tiny,
	numbersep=5pt,
	breaklines=true,
	breakautoindent=true, 
	breakindent=20pt,
	breakatwhitespace=true,
	showstringspaces=false,
	%prebreak=\raisebox{0ex}[0ex][0ex]{\ensuremath{\color{cRed}\rhookswarrow}},
	postbreak=\raisebox{0ex}[0ex][0ex]{\ensuremath{\color{cRed}\lhookrightarrow\space}},
	basicstyle=\ttfamily,
	identifierstyle=\color{black},
	keywordstyle=\color{cRed},
	commentstyle=\color{cGreen},
	stringstyle=\color{cOrange},
	morekeywords={uint8_t, int8_t, uint16_t, int16_t, uint32_t, int32_t, bool},
	emph=[1]
	{
		%enum list
		connectionTimeout, ObtCom_msgTimeout,
		stIdle, stWaitForConnection, 
		evTimeout, evNewReq,evNewMsg, evReceivedReq, evReceivedAck, evReceivedErr, evReceivedAns, evNewXXX, evReceivedXXX, 
		msg, req, ans, ack, err, ObtCom_numIds,
		CAL_USR_INP_REQ, CAL_FIN,
		STOP_FORWARD, STOP_BACKWARD, STOP_DIR_AUTO, NO_LIMIT, FIX_DIR, TURN_FIRST
	},
	emphstyle=[1]{\color{cBlue}\textit},
	emph=[2]
	{
		%typedef list
		error_t,
		ObtCom_msgId_t,	ObtCom_msgReaction_t, ObtCom_ansReaction_t, ObtCom_reqReaction_t, ObtCom_msg_t, ObtCom_msgType_t,
		canCom_msg_t, canCom_moduleAddr_t, canCom_subId_t,
		senstime_t, sensmodule_t,
		posCtrl_way_t, posCtrl_stopDir_t, posCtrl_wayLimit_t, posCtrl_dest_t, pos_t
	},
	emphstyle=[2]{\color{cGreen!70!black}},
	emph=[3]
	{
		%function list
		main,
		canCom_init,
		fillStackWithPattern,
		encodeId, decodeId, ObtCom_validateMessage,
		display_registerNewSettingScreen, display_newSubMenuEntryToggle, display_newSubMenuEntrySelection, display_newSubMenuEntryButton, display_newSubMenuEntryText, display_newSubMenuEntryNumberInput, display_editEntryText, display_setVisibilitySubMenuEntry, display_drawScreen, display_isSubMenuActive,
		runTest, makeMenu, switchMode, getMode, toggleDebugMode, getDebugMode,setMaxSpeed, getMaxSpeed,
		calloc, malloc,
		cal_setError, cal_start, cal_stop, cal_off, cal_getMode, cal_reset,
		atan2,
		posCtrl_addDest, posCtrl_clearAllDests, posCtrl_getDest, posCtrl_secStop
	},
	emphstyle=[3]{\color{cBlue!60!black}},
}

%use command const
\DeclareMathOperator{\const}{Konstant}

%title informations
\title[Eurobot] % (optional, only for long titles)
{Eurobot 2017 -- Moon Village}
\author[C. Angehrn \\M. Knöpfel \\T. Schneider] % (optional, for multiple authors)
{Cornel Angehrn \and Matthias Knöpfel \and Tibor Schneider}
\institute[hsr, imes] % (optional)
{
  HSR Hochschule für Technik Rapperswil \and
  IMES Institut für Mikroelektronik und Embedded Systems
}
%\date[KPT 2004] % (optional)
%{Conference on Presentation Techniques, 2004}
\subject{Embedded Systems}

%remove title and author from sidebar
\makeatletter
	\setbeamertemplate{sidebar \beamer@sidebarside}
	{
		\beamer@tempdim=\beamer@sidebarwidth
		\advance\beamer@tempdim by -6pt
		\insertverticalnavigation{\beamer@sidebarwidth}
		\vfill
		\ifx\beamer@sidebarside\beamer@lefttext
		\else
			\usebeamercolor{normal text}
			\llap{\usebeamertemplate***{navigation symbols}\hskip0.1cm}
			\vskip2pt
		\fi
	}
\setbeamertemplate{footline}
{
	\leavevmode%
	\hbox{%
		\begin{beamercolorbox}[wd=.333333\paperwidth,ht=2.25ex,dp=1ex,center]{title in head/foot}%
			\color{black} \insertshortauthor
		\end{beamercolorbox}%
		\begin{beamercolorbox}[wd=.333333\paperwidth,ht=2.25ex,dp=1ex,center]{title in head/foot}%
			\color{black} \insertshorttitle
		\end{beamercolorbox}%
		\begin{beamercolorbox}[wd=.333333\paperwidth,ht=2.25ex,dp=1ex,right]{title in head/foot}%
			\color{black} \insertshortdate{}\hspace*{2em}
			\color{black} \insertframenumber{} / \inserttotalframenumber\hspace*{2ex} 
	\end{beamercolorbox}}%
	\vskip0pt%
}
\patchcmd\beamer@@tmpl@frametitle{\insertframetitle}{\insertsection: \insertframetitle}{}{}
\makeatother


\begin{document}

  %titelseite
  \begingroup
  \makeatletter
  \setlength{\hoffset}{-.5\beamer@sidebarwidth}
  \makeatother
  \begin{frame}[plain]
  	\begin{figure}
	  	\includegraphics[height=1.4cm]{../images/hsrImesLogo.jpg}
  	\end{figure}
  \titlepage 
  \end{frame}
	\endgroup
  
  %HSR-Logo auf allen Seiten ab hier
  \addtobeamertemplate{frametitle}{}{%
  \begin{tikzpicture}[remember picture,overlay]
  \node[anchor=north west,yshift=-8pt, xshift=-6pt, xshift=1.5pt] at (current page.north west) {\includegraphics[height=0.8cm]{../images/HSR_Logo_A0.jpg}};
  \end{tikzpicture}} 

  %inhaltsverzeichnis
  %\frame{\frametitle{Inhaltsverzeichnis}\tableofcontents}
  
  %Einleitung
  \section{Einleitung}
\begin{frame}
	\frametitle{\,}
	
	\begin{figure}
	   	\centering
	   	\includegraphics[height = 5cm]{../images/presentation/LunarBaseESA.jpg}
	   	\caption{Quelle: http://www.esa.int/spaceinimages/Images}
	\end{figure}
	
	%Lunar base aus 3D-Druck-Teilen und Mondgestein,
	%ESA (European Space Ageny)-Projekt,
	%Eurobot 2017-Wettbewerb unter diesem Moto

\end{frame} 

\begin{frame}
	\frametitle{Projektteam}
	
	\begin{figure}
		\begin{tikzpicture}[scale=0.5, transform shape]
		%Joel
		\def\xPos{2};	\def\yPos{1};
		\draw [draw=none, fill=hsrBlue] (\xPos - 2,\yPos - 4) rectangle (\xPos + 2,\yPos + 1);
		\draw [white] (\xPos,\yPos + 0.5) node{Joel Stolz};
		\draw [white] (\xPos,\yPos) node{\textbf{Mechanik}};
		\node[inner sep=0, outer sep=0, align=center] at (\xPos, \yPos - 2.2) {\includegraphics[height=3.5cm]{../images/Projektorganisation/joelStolz.jpg}};
		%Tibor
		\def\xPos{7};	\def\yPos{1};
		\draw [draw=none, fill=hsrBlue] (\xPos - 2,\yPos - 4) rectangle (\xPos + 2,\yPos + 1);
		\draw [white] (\xPos,\yPos + 0.5) node{Tibor Schneider};
		\draw [white] (\xPos,\yPos - 0) node{\textbf{Elektronik}};
		\node[inner sep=0, outer sep=0, align=center] at (\xPos, \yPos - 2.2) {\includegraphics[height=3.5cm]{../images/Projektorganisation/tiborSchneider.jpg}};
		%cornel
		\def\xPos{12};	\def\yPos{1};
		\draw [draw=none, fill=hsrBlue] (\xPos - 2,\yPos - 4) rectangle (\xPos + 2,\yPos + 1);
		\draw [white] (\xPos,\yPos + 0.5) node{Cornel Angehrn};
		\draw [white] (\xPos,\yPos - 0) node{\textbf{Elektronik}};
		\node[inner sep=0, outer sep=0, align=center] at (\xPos, \yPos - 2.2) {\includegraphics[height=3.5cm]{../images/Projektorganisation/cornelAngehrn.jpg}};
		%Matthias
		\def\xPos{17};	\def\yPos{1};
		\draw [draw=none, fill=hsrBlue] (\xPos - 2,\yPos - 4) rectangle (\xPos + 2,\yPos + 1);
		\draw [white] (\xPos,\yPos + 0.5) node{Matthias Knöpfel};
		\draw [white] (\xPos,\yPos - 0) node{\textbf{Elektronik}};
		\node[inner sep=0, outer sep=0, align=center] at (\xPos, \yPos - 2.2) {\includegraphics[height=3.5cm]{../images/Projektorganisation/matthiasKnoepfel.jpg}};
		%Petra
		\def\xPos{22};	\def\yPos{5};
		\draw [draw=none, fill=hsrBlue] (\xPos - 2,\yPos - 4) rectangle (\xPos + 2,\yPos + 1);
		\draw [white] (\xPos,\yPos + 0.5) node{Petra Freuler};
		\draw [white] (\xPos,\yPos - 0) node{\textbf{Projektanalyse}};
		\node[inner sep=0, outer sep=0, align=center] at (\xPos, \yPos - 2.2) {\includegraphics[height=3.5cm]{../images/Projektorganisation/petraFreuler.jpg}};
		%Wüst
		\def\xPos{2};	\def\yPos{8};
		\draw [draw=none, fill=hsrBlue] (\xPos - 2.5,\yPos - 1) rectangle (\xPos + 2.5,\yPos + 1);
		\draw [white] (\xPos,\yPos + 0.25) node{Prof. Theodor Wüst};
		\draw [white] (\xPos,\yPos - 0.25) node{\textbf{Betreuung Mechanik}};
		%Brändle
		\def\xPos{11};	\def\yPos{8};
		\draw [draw=none, fill=hsrBlue] (\xPos - 2.5,\yPos - 1) rectangle (\xPos + 2.5,\yPos + 1);
		\draw [white] (\xPos,\yPos + 0.5) node{Prof. Erwin Brändle};
		\draw [white] (\xPos,\yPos) node{\textbf{Auftraggeber}};
		\draw [white] (\xPos,\yPos - 0.5) node{\textbf{Betreuung Elektronik}};
		%Keller
		\def\xPos{17};	\def\yPos{8};
		\draw [draw=none, fill=hsrBlue] (\xPos - 2.5,\yPos - 1) rectangle (\xPos + 2.5,\yPos + 1);
		\draw [white] (\xPos,\yPos + 0.5) node{Prof. Daniel Keller};
		\draw [white] (\xPos,\yPos) node{\textbf{Betreuung}};
		\draw [white] (\xPos,\yPos - 0.5 ) node{\textbf{Projektanalyse}};
		
		%connections
		\draw [thick] (2,2) -- (2,7) (2,3) -- (17,3) (7,2) -- (7,3) (12,2) -- (12,3) (17,2) -- (17,3) (11,3) -- (11,5) -- (20,5) (11,5) -- (11,7) (16,5) -- (17,5) -- (17,7);
		\end{tikzpicture}
	\end{figure}

\end{frame}

\begin{frame}
	\frametitle{Arbeitsaufteilung}
	Bereits während der Studienarbeit erledigt:
	\begin{itemize}
		\item Konzepte erstellt
		\item Gegnererkennung, Fahrcontroller und Softwareteile Mainboard vorbereitet
	\end{itemize}
	\vspace{1em}
	Aufgaben Bachelorarbeit:
	\begin{itemize}
	   	\item Fertigstellung der elektrischen Komponenten
	   	\item Strategie-Einheit, Wegfindung, \textit{collision handling}
	   	\item Zusammenbau und Test
	\end{itemize}

\end{frame}
  
  %Spielbeschreibung
  \section{Spielbeschreibung}

\begin{frame}
	\frametitle{Tisch und Spielelemente}
	\begin{figure}
		\centering
		\includegraphics[width = 13cm]{../images/presentation/spielfeldElemente.jpg}
	\end{figure}
	
\end{frame}

\begin{frame}
	\frametitle{Aufgaben und Punkte}
	
	\begin{figure}[H]
		\centering
		\begin{tabular}{|c|c|}
			\hline
			Ressourcen im Startfeld & 2 Punkte\\
			\hline
			\textit{Titanium Ores} in \textit{Cargo Bay} & 3 Punkte\\
			\hline
			\textit{Lunar Modules} in \textit{Moonbase} & 10 Punkte\\
			\hline
			\textit{Funny Action} & 20 Punkte\\
			\hline
			Bonus für aus dem Startfeld fahren & 15 Punkte\\ 
			\hline
			\hline
			\textit{Moon Rocks} & 0 Punkte\\
			\hline
			Strafe für unfaires und regelwidriges Verhalten & -20 Punkte\\
			\hline
		\end{tabular}	
	\end{figure}
\end{frame}

\begin{frame}
	
	%todo: Nicht benötigt!!
	
	\frametitle{Anforderungen an die Roboter}
	
	\begin{itemize}
		\item mechanische Abmessungen gemäss Reglement
		\item Roboter bewegen sich autonom
		\item keine Kollisionen mit Gegner
		\item keine gefährliche Techniken und Materialien (Pyrotechnik, starke Laser, ...)
	\end{itemize}
	
\end{frame}

\begin{frame}
	
	%todo Nicht benötigt
	
	\frametitle{Wettbewerb}
	
	\begin{itemize}
		\item Homologation
	\end{itemize}
	\begin{itemize}
		\item 3 min Vorbereitungszeit
		\item 90 s Spiel
		\item \textit{Funny Action}
	\end{itemize}
	\begin{itemize}
		\item Finale \textit{Best-of-three}
	\end{itemize}
\end{frame}
  
  %Roboterfunktionen
  \section{Roboterfunktionen}
\subsection{Grosser Roboter}

\begin{frame}
	\frametitle{Supportrad}
	\begin{columns}
		\begin{column}{0.42 \textwidth}
			\begin{itemize}
				\item nicht benötigt
			\end{itemize}
		\end{column}
		\begin{column}{0.58 \textwidth}
			\vspace{-2.8em}
			\begin{figure}[h]
				\centering
				\includegraphics[width = 1 \textwidth]{../images/presentation/supportrad.png}
			\end{figure}
		\end{column}
	\end{columns}
\end{frame}

\begin{frame}
	\frametitle{Walze und Leitplanke}
	\begin{columns}
		\begin{column}{0.6 \textwidth}
			\begin{itemize}
				\item Aufnahme der \textit{Titanium Ores}
				\item Spielelemente dürfen nicht beschäditgt werden
				\item Gummibänder und Kabelbinder
				\item Bereichsvergrösserung durch Leitplanken
			\end{itemize}
		\end{column}
		\begin{column}{0.4 \textwidth}
			\vspace{-2.5em}
			\begin{figure}[h]
				\centering
				\includegraphics[width = 0.9 \textwidth]{../images/presentation/walze.jpg}
			\end{figure}
			\vspace{-2.2em}
			\begin{figure}[h]
				\centering
				\includegraphics[width = 0.9 \textwidth]{../images/presentation/leitplanke.jpg}
			\end{figure}
		\end{column}
	\end{columns}
\end{frame}

\begin{frame}
	\frametitle{Riemen und Abstreifvorrichtung}
	\begin{columns}
	\begin{column}{0.6 \textwidth}
		\begin{itemize}
			\item Beförderung zum Schussmechanismus
			\item Klebeband
			\item mechanischer Abstreifer
			\item \textit{Titanium Ores} von \textit{Moon Rocks} trennen
		\end{itemize}
	\end{column}
	\begin{column}{0.4 \textwidth}
		\vspace{-2.5em}
		\begin{figure}[h]
			\centering
			\includegraphics[width = 0.9 \textwidth]{../images/presentation/riemen.jpg}
		\end{figure}
		\vspace{-2.2em}
		\begin{figure}[h]
			\centering
			\includegraphics[width = 0.9 \textwidth]{../images/presentation/abstreifvorrichtung.jpg}
		\end{figure}
	\end{column}
\end{columns}
\end{frame}

\begin{frame}
	\frametitle{Schussmechanismus}
		\begin{columns}
		\begin{column}{0.42 \textwidth}
			\begin{itemize}
				\item schiesst \textit{Titanium Ores} in die \textit{Cargo Bay}
				\item zwei sich drehende Rollen
				\item variable Distanz und Richtung
				\item Berechnung anhand der Position
			\end{itemize}
		\end{column}
		\begin{column}{0.58 \textwidth}
			\vspace{-2.8em}
			\begin{figure}[h]
				\centering
				\includegraphics[width = 1 \textwidth]{../images/presentation/schussmechanismus.jpg}
			\end{figure}
		\end{column}
	\end{columns}
\end{frame}

\subsection{Kleiner Roboter}

\begin{frame}
	\frametitle{Schere und Schranken}
	Mithilfe eines Scherenzugs sollten die \textit{Lunar Modules} aus den Raketen gezogen werden.
	Die Schranken verhindern ein umfallen der nachrutschenden \textit{Lunar Modules}.
	%Bilder
	Dieses Konzept funktionierte nicht wunschgemäss und wurde deshalb ersetzt.	
\end{frame}

\begin{frame}
	\frametitle{Greifer und Pressfinger}
	Greifer und Pressfinger wurden als Ersatz für Schere und Schranken entwickelt und konnten erfolgreich eingesetzt werden.
	%Bilder
\end{frame}

\begin{frame}
	\frametitle{Ring}
	Die \textit{Lunar Modules} haften während dem Transport an den Klebestellen des Rings.
	Durch drehen des Rings werden sie gekippt und abgestreift.
\end{frame}

\begin{frame}
	\frametitle{Drehmechanismus}
	Ein Arm mit einem Farbsensor und einem Rad kann ausgeklappt werden um \textit{Lunar Modules} zu drehen oder zu verschieben.
\end{frame}
  
  %Mainboard und Strategie-Einheit
  \section{Strategie-Einheit}

\begin{frame}
	\frametitle{Übersicht}
	
\end{frame}

\begin{frame}
	\frametitle{Aufgaben (\textit{Jobs})}
	
\end{frame}

\begin{frame}
	\frametitle{Pfadsuche und Spielfeld}
	
\end{frame}

\begin{frame}
	\frametitle{\textit{collision handling}}
	
\end{frame}

\begin{frame}
	\frametitle{Erkentnisse und Ergebnis}
	
\end{frame}
  
  %Fahrkontroller
  \section{Fahrcontroller}
\begin{frame}
\frametitle{Fahrcontroller}
\framesubtitle{Kalibrierung}

\begin{columns}
	\begin{column}{0.42 \textwidth}
		\begin{itemize}
			\item Berechnung automatisiert
			\item Bedienung vereinfacht
		\end{itemize}
	\end{column}
	\begin{column}{0.58 \textwidth}
		\vspace{-2.8em}
		\begin{figure}[h]
			\centering
			\includegraphics[width = 1 \textwidth]{../images/DC/calSMPres}
		\end{figure}
	\end{column}
\end{columns}

\end{frame}

\begin{frame}
\frametitle{Fahrcontroller}
\framesubtitle{Positionsregelung}

\begin{columns}
	\begin{column}{0.53 \textwidth}
		\begin{itemize}
			\item Regelung statt Steuerung
			\item Punkte müssen nicht genau angefahren werden
			\item Höhere Durchschnittsgeschwindigkeit
		\end{itemize}
	\end{column}
	\begin{column}{0.47 \textwidth}
		\vspace{-4em}
		\begin{figure}[h]
			\centering
			\includegraphics[width = 1.02 \textwidth]{../images/DC/PosControlPres.pdf}
		\end{figure}
	\end{column}
\end{columns}

\end{frame}

\begin{frame}
\frametitle{Fahrcontroller}
\framesubtitle{Demo}

\begin{columns}
	\begin{column}{0.4 \textwidth}
		\begin{itemize}
			\item Richtung wählbar
			\item Genauigkeit einstellbar
			\item verschiedene Modi
		\end{itemize}
	\end{column}
	\begin{column}{0.6 \textwidth}
		\begin{tikzpicture}[scale = 1.15, transform shape]	
			\draw [fill = hsrBlue] (0, 0) circle (2pt) node[above, color=black]{1,7};
			\draw [-, line width = 0.2mm, hsrBlue] (0, 0) -- (1, 0);
			\draw [fill = hsrBlue] (1, 0) circle (2pt) node[above, color=black]{2};
			\draw [-, line width = 0.2mm, hsrBlue] (1, 0) -- (2, 1);
			\draw [fill = hsrBlue] (2, 1) circle (2pt) node[above, color=black]{3};
			\draw [-, line width = 0.2mm, hsrBlue] (2, 1) -- (4, 1);
			\draw [fill = hsrBlue] (4, 1) circle (2pt) node[above, color=black]{4};
			\draw [-, line width = 0.2mm, hsrBlue] (4, 1) -- (5, 0);
			\draw [fill = hsrBlue] (5, 0) circle (2pt) node[above, color=black]{5};
			\draw [-, line width = 0.2mm, hsrBlue] (5, 0) -- (6, 0);
			\draw [fill = hsrBlue] (6, 0) circle (2pt) node[above, color=black]{6};
			\draw [-, line width = 0.2mm, hsrBlue] (5, 0) -- (3, 0);
			\draw [-, line width = 0.2mm, hsrBlue] (3, 0) -- (1, 0);
		\end{tikzpicture}
	\end{column}
\end{columns}

\end{frame}

  
  %Gegnererkennung
  \section{Gegnererkennung}
\begin{frame}
\frametitle{Gegnererkennung}
\framesubtitle{Komponenten}

\begin{figure}
	\begin{tikzpicture}[scale=1, transform shape]
	\def\height{4cm}
	\node[anchor=south west,inner sep=0] (image) at (0,0) {\includegraphics[height=\height] {../images/OD/Adapterboard4.JPG}};
	\begin{scope}[x={(image.south east)},y={(image.north west)}]
		\draw [line width=0.4mm, ->, cGreen] (0.35,1.1) node[left, color=black]{Treiberstufe} -- (0.35,0.75);  
		
		\draw [line width=0.4mm, ->, cGreen] (0.52,1.1) node[right, color=black]{Spannungsregler} -- (0.52, 0.5);
		\draw [line width=0.4mm, ->, cGreen] (0.62,-0.1) node[right, color=black]{CAN-Treiber} -- (0.62, 0.10);
		
		\draw [line width = 0.6mm, -, cRed] (0.15,0.5) to[out = 270, in = 180] (0.5, -0.2);
		\draw [line width = 0.6mm, -, cRed] (0.5, -0.2) -- (1.0, -0.2); 
		%\draw [line width = 0.6mm, -, cRed] (1.0, -0.2) to[out = 0, in = 270] (1.37, 0.1);
		
	\end{scope}
	
	\begin{scope}[xshift=6cm]
		\node[anchor=south west,inner sep=0] (rimage) at (0,0) {\includegraphics[height=\height] {../images/OD/Kopf3Pres.JPG}};
		\begin{scope}[x={(rimage.south east)},y={(rimage.north west)} ]
		
			\draw [line width=0.4mm, ->, cGreen] (0.5,1.02) node[above, color=black]{Motor und Spiegel} -- (0.5, 0.8);
			\draw [line width=0.4mm, ->, cGreen] (0.55,-0.1) node[right, color=black]{LED-Ring} -- (0.55, 0.4);
			
			\draw [line width = 0.6mm, -, cRed] (-0.32, -0.1985) to[out = 0, in = 270] (0.32, 0.1);
			\draw [line width = 0.4mm, -, cRed] (-0.49, 0.87) -- (-0.49, 0.44);
			\draw [line width = 0.6mm, -, cRed] (-0.49, 0.655) to[out = 0, in = 90] (-0.2, 0.44);
			\draw [line width = 0.6mm, -, cRed] (-0.2, 0.44) -- (-0.2, 0.0);
			\draw [line width = 0.6mm, -, cRed] (-0.2, 0.0) to[out = 270, in = 270] (0.5, 0.0);
		
		\end{scope}
	\end{scope}
	\end{tikzpicture}	
\end{figure}

\end{frame}

\begin{frame}
	\frametitle{Gegnererkennung}
	\framesubtitle{Ausmessung Spiegelhalter}
	
	\begin{figure}
		\centering
		\begin{tikzpicture}[y=-1cm, scale=0.5, transform shape]
		%near right
		\draw [draw=cRed, fill=clRed, fill opacity=0.3] (300:0.6) -- (300:1.95) -- (330:1.65) -- (0:1.6) -- (30:1.8) -- (60:2.4) -- (60:0.8) -- (30:0.65) -- (0:0.5) -- (330:0.5) -- (300:0.6);		
		%middle right
		\draw [draw=cBlue, fill=clBlue, fill opacity=0.3] (300:0.95) -- (330:0.93) -- (0:0.88) -- (30:0.85) -- (60:0.8) -- (60:2.75) -- (30:3.1) -- (0:3.05) -- (330:3.1) -- (300:3.05) -- (300:0.95);
		%middle left
		\draw [draw=cBlue, fill=clBlue, fill opacity=0.3] (120:0.73) -- (150:0.73) -- (180:0.75) -- (210:0.78) -- (240:0.88) -- (240:2.65) -- (210:2.8) -- (180:2.65) -- (150:2.35) -- (120:2.35) -- (120:0.73);
		%far right
		\draw [draw=cGreen, fill=clGreen, fill opacity=0.3] (300:1.25) -- (330:1.25) -- (0:1.1) -- (30:1) -- (60:0.88) -- (60:3.15) -- (30:4.2) -- (0:4.95) -- (330:4.9) -- (300:5) -- (300:1.5);
		
		%far left
		\draw [draw=cGreen, fill=clGreen, fill opacity=0.3] (120:0.65) -- (150:0.55) -- (180:0.6) -- (210:0.7) -- (240:0.83) -- (240:2.9) -- (210:2.25) -- (180:2) -- (150:1.75) -- (120:1.85) -- (120:0.65);
		%middle left
		\draw [draw=cBlue, fill=clBlue, fill opacity=0.3] (120:0.73) -- (150:0.73) -- (180:0.75) -- (210:0.78) -- (240:0.88) -- (240:2.65) -- (210:2.8) -- (180:2.65) -- (150:2.35) -- (120:2.35) -- (120:0.73);
		% near left
		\draw [draw=cRed, fill=clRed, fill opacity=0.3] (120:1.4) -- (150:1.7) -- (180:1.75) -- (210:1.65) -- (240:1.25) -- (240:4.25) -- (210:5) -- (180:5) -- (150:5) -- (120:5) -- (120:1.4);
		
		%draw Coordinate System
		\foreach \phi/\t in {0/90,30/120,60/150,90/180,120/210,150/240,180/270,210/300,240/330,270/0,300/30,330/60}
		{
			\draw [color=cGray] (0,0) -- (\phi:5.5);
			\node [fill=white] at (\phi:5.7) {\t°};
		}
		\foreach \r in {1,1.5,2,2.5,3,3.5,4,4.5,5}
		{
			\draw [color=cGray] (0,0) circle (\r);
		}
		\foreach \t/\x in {200/1,400/2,600/3,800/4,1000/5}
		{
			\node [fill=white, rounded corners=2mm] at (270:\x) {\t};
		}
		
		\draw [thick, ->] (0:7) -- (0:9) node[pos=0.5, above] {Vorwärts};
		
		\end{tikzpicture}
	\end{figure}

\end{frame}
  
  %Komponenten
  \section{Elektrische Komponenten}

\begin{frame}
	\frametitle{Servos und Smart Servos}
	
\end{frame}

\begin{frame}
	\frametitle{Dual Motor Controller}
	
\end{frame}

\begin{frame}
	\frametitle{Farbsensor}
	
\end{frame}

\begin{frame}
	\frametitle{Adapterboard zu Mainboard}
	
\end{frame}
  
  %Datenlogger
  \section{Datenlogger}

\begin{frame}
	\frametitle{Hardware}
	\begin{itemize}
		\item VDIP1-Board von FTDI
		\item Daten auf USB-Stick speichern und lesen
		\item UART-Schnittstelle
		\item einfache Befehle
	\end{itemize}
\end{frame}

\begin{frame}
	\frametitle{Software}
	\framesubtitle{Initialisierung}
	\begin{itemize}
		\item Reset
		\item Einstellungen
		\item Ordnerstruktur
		\item Log-Datei
	\end{itemize}
\end{frame}

\begin{frame}
	\frametitle{Software}
	\framesubtitle{Ereignisse loggen}
	\begin{itemize}
		\item Log-Eintrag
		\item Zwischenspeicher
		\item interne Zeit
	\end{itemize}
\end{frame}

\begin{frame}
	\frametitle{Software}
	\framesubtitle{Konfigurationsdaten}
	\begin{itemize}
		\item Zahlenwerte speichern und lesen
		\item CSV-Format
		\item Identifikation über Zeilennummer
		\item Beschreibung für Nutzer
		\item \textit{Callback}-Funktionen
	\end{itemize}
\end{frame}

\begin{frame}
	\frametitle{Probleme}
	\begin{itemize}
		\item Log-Datei wird nicht geschlossen
		\item Absturz des Loggers
	\end{itemize}
	
\end{frame}
  
  %Aufgaben und Strategie
  \section{Aufgaben und Strategie}

%todo vieleicht nur die strategie erklären, welche wir auch hauptzächlich gefahren sind. Eine Folie welche die aufgaben beschreibt.

\subsection{Grosser Roboter}

\begin{frame}
	\frametitle{Grosser Roboter}
	\framesubtitle{Aufgaben}
	\begin{itemize}
		\item Überquerung der Wippe
		\item Aufnahme der \textit{Titanium Ores}
		\item Abschuss der \textit{Titanium Ores} auf dem Spielfeld
		\item Schutz der \textit{Moonbase} %todo eher liegende Bälle vor Moonbase aufsammeln
	\end{itemize}
\end{frame}

\begin{frame}
	
	%todo weglassen! bereits alles in aufgaben erklärt
	
	\frametitle{Grosser Roboter}
	\framesubtitle{Strategien}
	\begin{itemize}
		\item möglichst viele \textit{Titanium Ores} sammeln
		\item in eigener Spielfeldhälfte bleiben
		\item Schutz der \textit{Moonbase}
		\item \textit{Funny Action}
	\end{itemize}
\end{frame}

\subsection{Kleiner Roboter}

\begin{frame}
	\frametitle{Kleiner Roboter}
	\framesubtitle{Aufgaben}
	\begin{itemize}
		\item Modulaufnahme auf dem Feld und aus den Raketen
		\item \textit{Lunar Modules} in \textit{Moonbase} legen
		\item mehrfarbige \textit{Lunar Modules} drehen
		\item \textit{Funny Action}
	\end{itemize}
\end{frame}

\begin{frame}
	
	%todo weglassen
	
	\frametitle{Kleiner Roboter}
	\framesubtitle{Strategien}
	Anforderungen und Varianten:
	\begin{itemize}
		\item so viele Punkte wie möglich erzielen %offensichtlich
		\item dem grossen Roboter aus dem Weg gehen
	\end{itemize}
	\begin{itemize}
		\item in eigener Spielfeldhälfte bleiben
		\item beim Gegner versuchen Punkte zu erzielen
		\item schnell mittlere \textit{Moonbase} mit eigenen \textit{Lunar Modules} füllen
	\end{itemize}
	
\end{frame}

  
  %Wettkampfanalyse
  \section{Wettkampfanalyse}

%auch die wurde weggelassen, vieleicht wenige Worte dazu verlieren
%\begin{frame}
%	\frametitle{Homologation}
%	
%\end{frame}

\begin{frame}
	\frametitle{Schweizermeisterschaft}
	
	\vspace{-3em}
	
	\begin{columns}[t]
		\begin{column}{0.45\textwidth}
			\begin{center}
				\begin{itemize}
					\item 14 Teams (12 Schweizer Teams)
					\item Probleme: ungenaues Fahren, Programmfehler, defektes Rad
					\item 4 Rang erreicht $\rightarrow$ Qualifikation
				\end{itemize}
			\end{center}
		\end{column}
		\begin{column}{0.55\textwidth}
			\begin{figure}
				\includegraphics[width=0.9\columnwidth]{../images/presentation/swisseurobot.jpg}
			\end{figure}
		\end{column}
	\end{columns}
	
\end{frame}

\begin{frame}
	\frametitle{Internationale Meisterschaft}
	
	\vspace{-3em}
	
	\begin{columns}[t]
		\begin{column}{0.45\textwidth}
			\begin{center}
				\begin{itemize}
					\item 28 Teams (3 besten Teams aus jedem Land)
					\item Probleme: Klima (Klebeband), Ungenauigkeiten des Spieltisches
					\item Viertelfinale erreicht $\rightarrow$ 8. Platz
				\end{itemize}
			\end{center}
		\end{column}
		\begin{column}{0.55\textwidth}
			\begin{figure}
				\includegraphics[width=0.9\columnwidth]{../images/presentation/international.jpg}
			\end{figure}
		\end{column}
	\end{columns}
	
\end{frame}
  
  %Rückblick und Fazit
  \section{Fazit}
\begin{frame}
\frametitle{Fazit}

\begin{itemize}
	\item Viele Ziele erreicht
	\item Einiges gelernt
	\item Sinnvolle Aufgabenteilung
	\item Harmonierendes Team
\end{itemize}
\end{frame}    
  
\end{document}
